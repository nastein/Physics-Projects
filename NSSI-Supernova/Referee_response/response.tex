\documentclass[aps,onecolumn,twoside,secnumarabic,balancelastpage,amsmath,amssymb,nofootinbib,hyperref=pdftex]{revtex4}


\usepackage{color}         % produces boxes or entire pages with colored backgrounds
\usepackage{graphics}      % standard graphics specifications
\usepackage[pdftex]{graphicx}      % alternative graphics specifications
\usepackage{longtable}     % helps with long table options
\usepackage[english]{babel}
\setlength{\parskip}{1em}
\usepackage{amsmath}
\usepackage{epsf}          % old package handles encapsulated post script issues
\usepackage{bm}            % special 'bold-math' package
\usepackage{verbatim}			% for comment environment
\usepackage[colorlinks=true]{hyperref}  % this package should be added after all others % use as follows: \url{http://web.mit.edu/8.13}                   
\renewcommand{\thesection}{\Roman{section}}                                  

\begin{document}
\title{Response to Referee }
\author         {Minjie Lei, Noah Steinberg, James D. Wells}
%\email          {nastein@umich.edu}
%\date{\today}
\affiliation{Leinweber Center for Theoretical Physics, University of Michigan, Ann Arbor}

\maketitle


1) Agreed, changed in paper to "when the neutrino number density is comparable to or greater than the nucleon number density"
\vskip 0.18in
2) \& 3) Added paragraph in section 4.2 that explains these points qualitatively as suggested. 

\vskip 0.18in
4) Done, figures added with extended x axis to R = 1000 km. 
\vskip 0.18in
5) Agreed, multiple paragraphs have been added to address the issue of IBD/ES discrimination. Additionally a new observable, which is approximately as good and is independent of the ability to discriminate between interaction type, has been added.
\vskip 0.18in
6) The referee brings up a good point. If the uncertainty in mean neutrino energy is large, then this could lead to an asymmetric relative increase or decrease in the number of events detected in each interaction channel. It is difficult to find the uncertainty in the mean neutrino energy as a function of time for the late time regime, though fortunately uncertainties across supernova models are small in this time window. Regardless, the new observable that we have added (see point 5) is independent of interaction type identification, thus this issue is no longer relevant. 
\vskip 0.18in
7) Errors quoted in the paper were taken from averaging over multiple supernova simulations with the same model. Now these errors have been replaced with normal $\sqrt{N}$ statistics. Error estimates happen to be numerically unchanged. 
\vskip 0.18in
8) Done, plots of lepton energy for IBD and ES for both hierarchies and standard, FP-NSSI, and FV-NSSI added in section 4.





\end{document}