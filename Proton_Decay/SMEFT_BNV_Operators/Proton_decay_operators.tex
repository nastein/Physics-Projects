\documentclass[aps,onecolumn,twoside,secnumarabic,balancelastpage,amsmath,amssymb,nofootinbib,hyperref=pdftex]{revtex4}


\usepackage{color}         % produces boxes or entire pages with colored backgrounds
\usepackage{graphics}      % standard graphics specifications
\usepackage[pdftex]{graphicx}      % alternative graphics specifications
\usepackage{longtable}     % helps with long table options
\usepackage{amsmath}
\usepackage{epsf}          % old package handles encapsulated post script issues
\usepackage{bm}            % special 'bold-math' package
\usepackage{verbatim}			% for comment environment
\usepackage[colorlinks=true]{hyperref}  % this package should be added after all others % use as follows: \url{http://web.mit.edu/8.13}                                    
                                  

\begin{document}
\title{}
\author         {Noah Steinberg}
\email          {nastein@umich.edu}
\date{\today}
\affiliation{University of Michigan - Physics}

\maketitle

\section{Dimension 6 operators for Baryon Number Violation in the Standard Model}

In this note we list the lowest dimension operators which contribute to baryon number violation using only standard model fields. The constraints on these operators are: invariance under $SU(3) \times SU(2)_{L} \times U(1)_{Y}$ and Lorentz invariance. We follow \cite{Alonso} for a list of these operators. 

\begin{enumerate}
\item $Q^{duql}_{prst} = (\bar{d}^{c\alpha}_{p}u^{\beta}_{r})(\bar{q}^{ci\gamma}_{s}l^{j}_{t})\epsilon_{\alpha\beta\gamma}\epsilon_{ij}$
\item $Q^{qque}_{prst} = (\bar{q}^{ci\alpha}_{p}q^{j\beta}_{r})(\bar{u}^{c\gamma}_{s}e_{t})\epsilon_{\alpha\beta\gamma}\epsilon_{ij}$
\item $Q^{qqql}_{prst} = (\bar{q}^{ci\alpha}_{p}q^{j\beta}_{r})(\bar{q}^{ck\gamma}_{s}l^{l}_{t})\epsilon_{\alpha\beta\gamma}\epsilon_{il}\epsilon_{jk}$
\item $Q^{duue}_{prst} = (\bar{d}^{c\alpha}_{p}u^{\beta}_{r})(\bar{u}^{c\gamma}_{s}e_{t})\epsilon_{\alpha\beta\gamma}$
\end{enumerate}

In the above we have used 4 component Dirac notation. d,u, and e refer to right handed down and up type quark fields and charged leptons, l and q are left handed SU(2) quark and lepton doublets. The superscript '$c$' on a field like $\Psi^{c}$ denotes the charge conjugated field operator. SU(3) indices are $\alpha, \beta$, and $\gamma$, SU(2) indices are i, j, k, and l. Flavor indices are p, r, s, and t. To identify which combination of operators are relevant to the proton decay channel $p \rightarrow K^{+}\bar{\nu}$, we need just fill in the flavor structure (p, r, s, and t) from the four fermion operators above so that they contain a strange, up, and down quark, as well as a neutrino. It is easy to see that $O^{qque}_{prst}$ and $O^{duue}_{prst}$ do not contribute to this channel because they cannot contain a neutrino, assuming that the right handed neutrino does not exist. 


Overall we find that the operators $O^{duql}_{prst}$ and $O^{qqql}_{prst}$ together contribute 6 terms that will allow the transition $p \rightarrow K^{+}\bar{\nu}$. In the following we will leave the flavor index on the lepton doublet, t, arbitrary and $e^{-}$ and $\nu$ stand for any lepton doublet (electron, muon tau). From $O^{duql}_{prst}$ we find two terms:

\begin{enumerate}
\item $Q^{duql}_{112t} = \epsilon_{\alpha\beta\gamma}(\bar{d}_{R}^{c\alpha}u_{R}^{\beta})(\bar{c}_{L}^{c\gamma}e_{L}^{-} - \bar{s}_{L}^{c\gamma}\nu_{L})$
\item $Q^{duql}_{211t} = \epsilon_{\alpha\beta\gamma}(\bar{s}_{R}^{c\alpha}u_{R}^{\beta})(\bar{u}_{L}^{c\gamma}e_{L}^{-} - \bar{d}_{L}^{c\gamma}\nu_{L})$
\end{enumerate}
 
The above operators contribute because they have $sud\nu$ terms. The operator $O^{qqql}_{prst}$ contributes 4 terms:

\begin{enumerate}
\item $Q^{qqql}_{112t} = \epsilon_{\alpha\beta\gamma}[(\bar{u}_{L}^{c\alpha}u_{L}^{\beta})(\bar{s}_{L}^{c\gamma}e_{L}^{-}) - (\bar{u}_{L}^{c\alpha}d_{L}^{\beta})(\bar{c}_{L}^{c\gamma}e_{L}^{-}) - (\bar{d}_{L}^{c\alpha}u_{L}^{\beta})(\bar{s}_{L}^{c\gamma}\nu_{L}) + (\bar{d}_{L}^{c\alpha}d_{L}^{\beta})(\bar{c}_{L}^{c\gamma}\nu_{L})]$

\item $Q^{qqql}_{121t} = \epsilon_{\alpha\beta\gamma}[(\bar{u}^{c\alpha}_{L}c^{\beta}_{L})(\bar{d}^{c\gamma}_{L}e^{-}_{L}) - (\bar{u}^{c\alpha}_{L}s^{\beta}_{L})(\bar{u}^{c\gamma}_{L}e^{-}_{L}) - (\bar{d}^{c\alpha}_{L}c^{\beta}_{L})(\bar{d}^{c\gamma}_{L}\nu_{L}) + (\bar{d}^{c\alpha}_{L}s^{\beta}_{L})(\bar{u}^{c\gamma}_{L}\nu_{L})]$

\item $Q^{qqql}_{211t} =  \epsilon_{\alpha\beta\gamma}[(\bar{c}^{c\alpha}_{L}u^{\beta}_{L})(\bar{d}^{c\gamma}_{L}e^{-}_{L}) - (\bar{c}^{c\alpha}_{L}d^{\beta}_{L})(\bar{u}^{c\gamma}_{L}e^{-}_{L}) - (\bar{s}^{c\alpha}_{L}u^{\beta}_{L})(\bar{d}^{c\gamma}_{L}\nu_{L}) + (\bar{s}^{c\alpha}_{L}d^{\beta}_{L})(\bar{u}^{c\gamma}_{L}\nu_{L})]$

\end{enumerate}

From each term we can identify an $sdu\nu$ term which all together leads to the effective lagrangian

\begin{align}
\mathcal{L}_{p^{+}\rightarrow K^{+}\nu} = \epsilon_{\alpha\beta\gamma}[&-c^{duql}_{112t}(\bar{d}_{R}^{c\alpha}u_{R}^{\beta})(\bar{s}_{L}^{c\gamma}\nu_{L}) - c^{duql}_{211t}(\bar{s}_{R}^{c\alpha}u_{R}^{\beta})(\bar{d}_{L}^{c\gamma}\nu_{L}) - c^{qqql}_{112t}(\bar{d}_{L}^{c\alpha}u_{L}^{\beta})(\bar{s}_{L}^{c\gamma}\nu_{L})\nonumber \\
&+ c^{qqql}_{121t}(\bar{d}^{c\alpha}_{L}s^{\beta}_{L})(\bar{u}^{c\gamma}_{L}\nu_{L}) + c^{qqql}_{211t}(\bar{s}^{c\alpha}_{L}d^{\beta}_{L})(\bar{u}^{c\gamma}_{L}\nu_{L}) - c^{qqql}_{211t}(\bar{s}^{c\alpha}_{L}u^{\beta}_{L})(\bar{d}^{c\gamma}_{L}\nu_{L})]
\end{align}

Note in the above that $Q^{qqql}_{211t}$ contributes two terms to $p\rightarrow K^{+}\bar{\nu}$.
 
\begin{thebibliography}{9}

\bibitem{Alonso}
R. Alonso et al.,  "Renormalization group evolution of dimension-six
baryon number violating operators", JHEP 1404 (2014) 159 arXiv:1312.2014 [hep-ph] CERN-PH-TH-2013-305

\end{thebibliography}

\end{document}