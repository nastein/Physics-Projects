\documentclass[aps,onecolumn,twoside,secnumarabic,balancelastpage,amsmath,amssymb,nofootinbib,hyperref=pdftex]{revtex4}


\usepackage{color}         % produces boxes or entire pages with colored backgrounds
\usepackage{graphics}      % standard graphics specifications
\usepackage[pdftex]{graphicx}      % alternative graphics specifications
\usepackage{longtable}     % helps with long table options
\usepackage[english]{babel}
\setlength{\parskip}{1em}
\usepackage{amsmath}
\usepackage{epsf}          % old package handles encapsulated post script issues
\usepackage{bm}            % special 'bold-math' package
\usepackage{verbatim}			% for comment environment
\usepackage[colorlinks=true]{hyperref}  % this package should be added after all others % use as follows: \url{http://web.mit.edu/8.13}                                    
                                  

\begin{document}
\title{}
\author         {Noah Steinberg}
\email          {nastein@umich.edu}
\date{\today}
\affiliation{University of Michigan - Physics}

\maketitle

\section{Operator Contributions to Proton Decay Observables}

Below we list several proton decay channels which are potentially observable at current and future experiments. Alongside, we list the flavor structure of the contributions (if they exist) of the four possible SMEFT baryon number violating operators to the corresponding channel.

\begin{center}
\begin{tabular}{ | | c | c | c | c | c | |}
\hline
Observable & $Q^{duql}_{prst}$ & $Q^{qque}_{prst}$ & $Q^{qqql}_{prst}$ & $Q^{duue}_{prst}$ \\
\hline\hline
$p\rightarrow K^{+}\bar{\nu}_{t}$ & 112t, 211t & & 112t, 121t, 211t & \\
\hline
$p\rightarrow e^{+}\pi^{0}$ & 1111 & 1111 & 1111 & 1111 \\
\hline
$p\rightarrow \mu^{+}\pi^{0}$ & 1112 & 1112 & 1112 & 1112 \\
\hline
$p\rightarrow \mu^{+}K^{0}$ & 2112 & 1212, 2112 & 1212, 1122 & 2112 \\
\hline
$p\rightarrow e^{+}\eta$ & 1111 & 1111 & 1111 & 1111 \\
\hline
$p\rightarrow \mu^{+}\eta$ & 1112 & 1112 & 1112 & 1112 \\
\hline
$p\rightarrow e^{+}\rho^{0}$  & 1111 & 1111 & 1111 & 1111 \\
\hline
$p\rightarrow \mu^{+}\rho^{0}$ & 1112 & 1112 & 1112 & 1112 \\
\hline
$p\rightarrow e^{+}\omega^{0}$  & 1111 & 1111 & 1111 & 1111 \\
\hline
$p\rightarrow \mu^{+}\omega^{0}$ & 1112 & 1112 & 1112 & 1112 \\
\hline
$p\rightarrow \pi^{+}\bar{\nu}_{t}$ & 111t & & 111t & \\
\hline
$n\rightarrow K^{0}\bar{\nu}_{t}$ & 112t, 211t & & 112t, 121t, 211t &\\ 
\hline
$n\rightarrow e^{+}\pi^{-}$ & 1111 & 1111 & 1111 & 1111 \\
\hline
$n\rightarrow \mu^{+}\pi^{-}$ & 1112 & 1112 & 1112 & 1112 \\
\hline
$n\rightarrow e^{+}\rho^{-}$ & 1111 & 1111 & 1111 & 1111 \\
\hline
$n\rightarrow \mu^{+}\rho^{-}$ & 1112 & 1112 & 1112 & 1112 \\
\hline
$n\rightarrow \pi^{0}\bar{\nu}_{t}$ & 111t & 111t & 111t & 111t \\
\hline
\end{tabular}
\end{center}

Note that in nucleon decays resulting in a neutrino, all experiments are insensitive to the neutrino flavor, so it is left unspecified, denote by the flavor index "t". This list does not cover every single proton decay mode which is searched, but contains many of the most important decay channels. Additional channels can easily be added to this list.

It is interesting to note that in a supersymmetric theory with the dimension 5 BNV operators

\begin{equation}
\mathcal{L}^{5} = \int d^{2}\theta \frac{1}{2}\epsilon_{abc}(Q^{a}_{i}Q^{b}_{j})(Q^{c}_{k}L_{l}) + \int d^{2}\theta \epsilon^{abc}\bar{U}_{ia}\bar{E}_{j}\bar{U}_{kb}\bar{D}_{lc},
\end{equation}

many of the flavor combinations of operators listed above vanish. This is because the antisymmetry of the above lagrangian disallows terms where all quark flavors are equal. This is why low energy effective theories which are generated by integrating out sparticles from [above] have proton decay modes involving a strange quark.

\section{Discussion}

Above we listed some results for proton decay observables and the dimension 6 SMEFT operators that produce them. In the context of a UV SUSY theory it is hard to connect many of the operators. For example, those operators which have only one quark flavor present cannot be produced by dimension 5 SUSY BNV operators, but they can be produced by dimension 6 operators. 








\end{document}